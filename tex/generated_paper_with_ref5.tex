
\documentclass[12pt,letterpaper]{article}
\usepackage[utf8]{inputenc}
\usepackage{amsmath}
\usepackage{graphicx}
\usepackage[margin=1in]{geometry}
\usepackage{natbib}
\usepackage{booktabs}
\usepackage{float}

\title{DLCRec: A Novel Framework for Diversity Control in Large Language Model-Based Recommender Systems}
\author{Emma J. Thompson \and Rajesh K. Patel \and Sophia L. Chen \and Michael A. Rodriguez}

\begin{document}

\maketitle

\begin{abstract}
Large Language Models (LLMs) have shown remarkable potential in recommendation systems, but managing diversity while maintaining accuracy remains a significant challenge. This paper introduces DLCRec, a novel framework for fine-grained diversity control in LLM-based recommender systems. DLCRec employs task decomposition, data augmentation, and a control mechanism to achieve precise diversity management. We evaluate DLCRec on two real-world datasets, demonstrating its superior performance over state-of-the-art baselines across various recommendation scenarios. Our results show that DLCRec achieves the lowest MAE_Cov@10 in all scenarios, indicating excellent diversity control with minimal impact on accuracy. Through ablation studies and sensitivity analysis, we validate the effectiveness of our task decomposition and data augmentation strategies. This work presents a promising approach for developing more controllable and diverse LLM-based recommendation systems.
\end{abstract}

\section{Introduction}

Large Language Models (LLMs) have revolutionized various domains of artificial intelligence, including recommendation systems. Their ability to understand and generate human-like text has opened new avenues for creating more personalized and context-aware recommendations. However, a persistent challenge in recommendation systems is balancing accuracy with diversity.

Diversity in recommendations is crucial for several reasons: it helps prevent filter bubbles, improves user satisfaction, and increases the likelihood of serendipitous discoveries. However, naively increasing diversity often comes at the cost of recommendation accuracy. This trade-off presents a significant challenge, especially in LLM-based systems where the internal workings are often opaque.

In this paper, we introduce DLCRec, a novel framework designed to address the challenge of diversity control in LLM-based recommender systems. DLCRec employs a unique approach that decomposes the recommendation task into sub-tasks, utilizes data augmentation to overcome data scarcity, and implements a control mechanism for fine-grained diversity management.

Our main contributions are as follows:
1. We propose a task decomposition strategy that breaks down the recommendation process into genre prediction, genre filling, and item prediction sub-tasks.
2. We introduce data augmentation techniques specifically designed to address the scarcity of diversity-related user behavior data.
3. We develop a control framework that uses control numbers to guide the outputs of each sub-task, enabling precise diversity management.
4. We conduct extensive empirical evaluations on two real-world datasets, demonstrating the effectiveness of DLCRec in various recommendation scenarios.

\section{Methodology}

Our proposed DLCRec framework consists of three main components: task decomposition, data augmentation, and diversity control mechanism. Each component is designed to address specific challenges in managing diversity in LLM-based recommender systems.

2.1 Task Decomposition
We decompose the recommendation task into three sub-tasks:
1. Genre Prediction: Predicting the genres that a user is likely to be interested in.
2. Genre Filling: Determining the proportion of items from each predicted genre.
3. Item Prediction: Selecting specific items within each genre.

2.2 Data Augmentation
To address the scarcity of diversity-related user behavior data, we employ two data augmentation techniques:

1. Genre-based Item Swapping: We create synthetic user histories by swapping items within the same genre across different users' histories.
2. Diversity-oriented Negative Sampling: We generate negative samples by selecting items from genres that are underrepresented in a user's history.

2.3 Diversity Control Mechanism
We introduce a control number c ∈ [1, 10] to guide the diversity of recommendations. This control number influences each sub-task as follows:

1. Genre Prediction: c affects the number of genres predicted.
2. Genre Filling: c influences the distribution of items across genres.
3. Item Prediction: c guides the selection of items within each genre.

[IMAGE PLACEHOLDER 1] (chart)
[data_prompt] The data shows the impact of the control number on genre diversity and recommendation accuracy. The data has 3 columns labeled 'Control Number', 'Genre Diversity', and 'Recommendation Accuracy'. The values should show a trade-off between diversity and accuracy as the control number increases from 1 to 10.
[plot_prompt] Line chart with two y-axes. The left y-axis represents Genre Diversity (0-1 scale) with a blue line, and the right y-axis represents Recommendation Accuracy (0-100% scale) with a red line. The x-axis shows the Control Number from 1 to 10. Include markers for each data point and a legend labeling the two lines.

Figure 1: Impact of Control Number on Genre Diversity and Recommendation Accuracy

\section{Results}

We evaluated DLCRec on two real-world datasets: MovieLens10M and Steam. We compared our approach with several state-of-the-art baselines, including LLM-Rec, DiverRec, and ControlRec. Our evaluation metrics include Mean Average Error (MAE), Normalized Discounted Cumulative Gain (NDCG), and MAE_Cov@10 for diversity control.

3.1 Overall Performance
Table 1 presents the overall performance of DLCRec compared to the baselines on both datasets.

\begin{table}[h]
\centering
\begin{tabular}{|l|c|c|c|c|c|c|}
\hline
\multirow{2}{*}{Model} & \multicolumn{3}{c|}{MovieLens10M} & \multicolumn{3}{c|}{Steam} \\
\cline{2-7}
 & MAE & NDCG@10 & MAE_Cov@10 & MAE & NDCG@10 & MAE_Cov@10 \\
\hline
LLM-Rec & 0.721 & 0.892 & 0.315 & 0.684 & 0.876 & 0.298 \\
DiverRec & 0.735 & 0.885 & 0.287 & 0.701 & 0.869 & 0.263 \\
ControlRec & 0.728 & 0.888 & 0.256 & 0.692 & 0.872 & 0.241 \\
DLCRec & \textbf{0.719} & \textbf{0.895} & \textbf{0.213} & \textbf{0.681} & \textbf{0.879} & \textbf{0.205} \\
\hline
\end{tabular}
\caption{Overall performance comparison (lower MAE and MAE_Cov@10, higher NDCG@10 are better)}
\end{table}

DLCRec outperforms all baselines across both datasets, achieving the lowest MAE and MAE_Cov@10, and the highest NDCG@10. This indicates that our approach successfully balances accuracy and diversity control.

3.2 Ablation Study
We conducted an ablation study to evaluate the contribution of each component in DLCRec. Table 2 shows the results on the MovieLens10M dataset.

\begin{table}[h]
\centering
\begin{tabular}{|l|c|c|c|}
\hline
Model Variant & MAE & NDCG@10 & MAE_Cov@10 \\
\hline
DLCRec (Full) & \textbf{0.719} & \textbf{0.895} & \textbf{0.213} \\
w/o Task Decomposition & 0.731 & 0.887 & 0.245 \\
w/o Data Augmentation & 0.725 & 0.891 & 0.229 \\
w/o Control Mechanism & 0.722 & 0.893 & 0.238 \\
\hline
\end{tabular}
\caption{Ablation study results on MovieLens10M}
\end{table}

3.3 Sensitivity Analysis
We analyzed the sensitivity of DLCRec to different control numbers. Figure 2 illustrates the trade-off between recommendation accuracy (NDCG@10) and diversity (measured by the Gini coefficient) as the control number varies.

[IMAGE PLACEHOLDER 2] (chart)
[data_prompt] The data shows the relationship between the control number, NDCG@10, and Gini coefficient for diversity. The data has 3 columns labeled 'Control Number', 'NDCG@10', and 'Gini Coefficient'. The control number ranges from 1 to 10, NDCG@10 ranges from 0.8 to 0.9, and the Gini coefficient ranges from 0.3 to 0.7.
[plot_prompt] Scatter plot with two y-axes. The left y-axis represents NDCG@10 (0.8-0.9 scale) with blue circular markers, and the right y-axis represents the Gini Coefficient (0.3-0.7 scale) with red square markers. The x-axis shows the Control Number from 1 to 10. Include a legend labeling the two sets of markers and trend lines for both sets of data points.

Figure 2: Sensitivity Analysis - Impact of Control Number on Accuracy and Diversity

\section{Discussion}

Our results demonstrate that DLCRec successfully addresses the challenge of managing diversity in LLM-based recommender systems. The superior performance of DLCRec across different datasets and metrics suggests that our approach is both effective and generalizable.

The ablation study results highlight the importance of task decomposition in our framework. By breaking down the recommendation process into genre prediction, genre filling, and item prediction, DLCRec can better manage the diversity-accuracy trade-off. This decomposition allows for more fine-grained control over the recommendation process, enabling the system to adjust diversity at multiple levels.

The effectiveness of our data augmentation techniques is evident from the ablation study results. By addressing the scarcity of diversity-related user behavior data, these techniques improve the model's ability to generate diverse recommendations without compromising accuracy.

The sensitivity analysis results demonstrate the flexibility of our control mechanism. By adjusting the control number, we can achieve a wide range of diversity levels while maintaining relatively stable accuracy. This feature is particularly valuable in real-world applications where different users or contexts may require varying levels of recommendation diversity.

Limitations and future work include addressing computational complexity, conducting user studies, incorporating temporal dynamics, improving explainability, and extending the framework to handle multiple dimensions of diversity simultaneously.

\section{Conclusion}

In this paper, we introduced DLCRec, a novel framework for managing diversity in LLM-based recommender systems. By employing task decomposition, data augmentation, and a control mechanism, DLCRec achieves fine-grained control over recommendation diversity while maintaining high accuracy. Our extensive empirical evaluation on two real-world datasets demonstrates that DLCRec outperforms state-of-the-art baselines across multiple recommendation scenarios.

The key contributions of this work include a task decomposition strategy, data augmentation techniques, and a flexible control mechanism. The success of DLCRec opens up several exciting avenues for future research in controllable recommendation systems, including extending the framework to handle multiple control dimensions, investigating its applicability to other domains, and developing more sophisticated control mechanisms.

\bibliographystyle{plain}
\begin{thebibliography}{99}

\bibitem{ref1}
Zhang, Y., et al. (2023). A Survey of Large Language Models for Recommendation. arXiv preprint arXiv:2305.19860.

\bibitem{ref2}
Kunaver, M., & Požrl, T. (2017). Diversity in recommender systems – A survey. Knowledge-Based Systems, 123, 154-162.

\bibitem{ref3}
Castells, P., et al. (2015). Novelty and diversity in recommender systems. In Recommender Systems Handbook (pp. 881-918). Springer, Boston, MA.

\end{thebibliography}

\end{document}