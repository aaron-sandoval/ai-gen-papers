
\documentclass[12pt,letterpaper]{article}
\usepackage[utf8]{inputenc}
\usepackage{amsmath}
\usepackage{graphicx}
\usepackage[margin=1in]{geometry}
\usepackage{natbib}
\usepackage{booktabs}
\usepackage{float}

\title{DLCRec: A Diversity-Controlled Large Language Model for Recommender Systems}
\author{Sophia Chen \and Rajiv Patel \and Emma Thompson \and Daniel Lee}

\begin{document}

\maketitle

\begin{abstract}
Large Language Models (LLMs) have shown great potential in recommender systems, but managing diversity while maintaining accuracy remains a challenge. This paper introduces DLCRec, a novel approach that enables fine-grained control over diversity in LLM-based recommender systems. DLCRec decomposes the recommendation task into genre prediction, genre filling, and item prediction sub-tasks, and employs data augmentation techniques to address the scarcity of diversity-related user behavior data. We propose a control framework that uses control numbers to guide the outputs of each sub-task. Extensive experiments on two real-world datasets demonstrate that DLCRec outperforms state-of-the-art baselines across multiple recommendation scenarios, achieving precise diversity control with minimal sacrifice in accuracy. Our approach shows promise for fine-tuning LLMs for controllable recommendations beyond diversity management.
\end{abstract}

\section{Introduction}

Recommender systems have become integral to various online platforms, helping users navigate vast amounts of content and make informed decisions. With the advent of Large Language Models (LLMs), there has been a significant shift in the capabilities and potential of these systems. However, one persistent challenge in recommender systems is balancing recommendation accuracy with diversity.

Diversity in recommendations is crucial for several reasons: it helps prevent filter bubbles, improves user satisfaction, and increases the likelihood of serendipitous discoveries. However, naively increasing diversity often comes at the cost of reduced accuracy, leading to less relevant recommendations.

In this paper, we introduce DLCRec, a novel approach for managing diversity in LLM-based recommender systems. DLCRec enables fine-grained control over diversity while maintaining high recommendation accuracy. Our key contributions are:

1. A task decomposition strategy that breaks down the recommendation process into genre prediction, genre filling, and item prediction sub-tasks.
2. Data augmentation techniques to address the scarcity of diversity-related user behavior data.
3. A control framework that uses control numbers to guide the outputs of each sub-task, allowing for precise diversity management.

We evaluate DLCRec on two real-world datasets and demonstrate its superiority over state-of-the-art baselines across multiple recommendation scenarios. Our results show that DLCRec achieves precise control over diversity with only marginal sacrifices in accuracy, paving the way for more balanced and effective recommender systems.

\section{Methodology}

Our proposed DLCRec framework consists of three main components: task decomposition, data augmentation, and a control mechanism for diversity management.

\textbf{Task Decomposition:}
We decompose the recommendation task into three sub-tasks:
1. Genre Prediction: Predicting the distribution of genres for a user's recommendations.
2. Genre Filling: Determining the number of items for each predicted genre.
3. Item Prediction: Selecting specific items within each genre.

This decomposition allows for more granular control over the diversity of recommendations. The genre prediction task is formulated as a multi-label classification problem.

\textbf{Data Augmentation:}
To address the scarcity of diversity-related user behavior data, we employ two data augmentation techniques:
1. Genre-based item swapping: Randomly replacing items in a user's history with items of the same genre.
2. Synthetic diverse user generation: Creating artificial user profiles with deliberately diverse item histories.

\textbf{Control Mechanism:}
We introduce a control number $c \in [1, 10]$ to guide the diversity of recommendations. This number influences each sub-task as follows:
1. Genre Prediction: $c$ affects the threshold for including a genre in the recommendations.
2. Genre Filling: $c$ determines the distribution of items across genres.
3. Item Prediction: $c$ influences the selection of items within each genre.

\section{Results}

We evaluated DLCRec on two real-world datasets: MovieLens10M and Steam. We compared our approach against several state-of-the-art baselines, including BERT4Rec, SASRec, and DiverseRec. Our evaluation metrics include NDCG@K, Precision@K, and MAE_Cov@K for measuring recommendation accuracy and diversity.

DLCRec outperforms all baselines in terms of NDCG@10 and Precision@10, indicating superior recommendation accuracy. More importantly, DLCRec achieves the lowest MAE_Cov@10, demonstrating its ability to precisely control diversity.

We conducted ablation studies to validate the importance of each component in DLCRec. The studies demonstrate that all components contribute significantly to DLCRec's performance. Notably, removing task decomposition led to the largest drop in both accuracy and diversity control.

Finally, we performed a sensitivity analysis to evaluate DLCRec's ability to generate recommendations with varying diversity levels. We observed that DLCRec maintains stable accuracy across different control numbers (1-10), while effectively adjusting the diversity of recommendations.

\section{Conclusion}

In this paper, we introduced DLCRec, a novel approach for managing diversity in Large Language Model (LLM)-based recommender systems. By leveraging task decomposition, data augmentation, and a control mechanism, DLCRec achieves fine-grained control over diversity while maintaining high recommendation accuracy.

Our extensive empirical evaluation on two real-world datasets demonstrates that DLCRec outperforms state-of-the-art baselines across multiple recommendation scenarios. The framework's ability to generate recommendations with varying diversity levels while maintaining stable accuracy showcases its potential for real-world applications where adaptability is crucial.

Future research directions include extending the framework to optimize list-level recommendations, exploring applications beyond diversity control, investigating transfer learning techniques, and conducting user studies to evaluate perceived quality and satisfaction of DLCRec-generated recommendations.

\bibliographystyle{plain}
\begin{thebibliography}{99}

\bibitem{zhang2023generative}
title={Generative Recommendation: Towards Next-generation Recommender Paradigm},
  author={Zhang, Shuyuan and Yao, Lina and Tay, Yi and Xu, Lina and Zhu, Xiwei and Bing, Lidong},
  journal={arXiv preprint arXiv:2304.03516},
  year={2023}


\bibitem{kunaver2017diversity}
title={Diversity in recommender systems--A survey},
  author={Kunaver, Matevz and Pozrl, Tomaz},
  journal={Knowledge-Based Systems},
  volume={123},
  pages={154--162},
  year={2017},
  publisher={Elsevier}


\bibitem{cenikj2021serendipity}
title={Serendipity in recommender systems: An overview and synthesis of the literature},
  author={Cenikj, Gjorgjina and Gievska, Sonja},
  journal={Information Processing & Management},
  volume={58},
  number={6},
  pages={102722},
  year={2021},
  publisher={Elsevier}


\bibitem{sun2019bert4rec}
title={BERT4Rec: Sequential recommendation with bidirectional encoder representations from transformer},
  author={Sun, Fei and Liu, Jun and Wu, Jian and Pei, Changhua and Lin, Xiao and Ou, Wenwu and Jiang, Peng},
  booktitle={Proceedings of the 28th ACM International Conference on Information and Knowledge Management},
  pages={1441--1450},
  year={2019}


\bibitem{kang2018self}
title={Self-attentive sequential recommendation},
  author={Kang, Wang-Cheng and McAuley, Julian},
  booktitle={2018 IEEE International Conference on Data Mining (ICDM)},
  pages={197--206},
  year={2018},
  organization={IEEE}


\bibitem{li2021diverserec}
title={DiverseRec: A Novel Framework for Improving Diversity in Sequential Recommendation},
  author={Li, Chenxiao and Wang, Weizhi and Shang, Jingjing and Zhang, Peng and Tang, Ruiming},
  booktitle={Proceedings of the 44th International ACM SIGIR Conference on Research and Development in Information Retrieval},
  pages={1863--1867},
  year={2021}


\end{thebibliography}

\end{document}